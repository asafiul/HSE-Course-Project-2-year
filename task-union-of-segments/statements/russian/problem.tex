\begin{problem}{Объединение отрезков}{стандартный ввод}{стандартный вывод}{2 секунды}{256 мегабайт}

На числовой прямой расположены $n$ отрезков, каждый из которых задан парой своими концами $[l_i, r_i]$ $(l_i < r_i)$. Отрезки не пересекаются и не имеют общих точек.

Разрешается не более $k$ раз выбрать один из отрезков и расширить его на 1 единицу влево или вправо. Если после расширения два отрезка имеют общую точку, то они объединяются в один.

Необходимо определить минимальное количество отрезков, которые могут остаться после совершения не более $k$ операций расширения.

\InputFile
Первая строка содержит два целых числа $n$ и $k$ ($1 \leq n \leq 2\cdot 10^5$, $1 \leq k \leq 10^9$)~--- количество отрезков и доступных операций расширения.

Каждая из следующих $n$ строк содержит пару целых чисел $l_i$ и $r_i$ ($0 \leq l_i < r_i \leq 10^9$)~--- концы $i$-го отрезка. Гарантируется, что для любых $i \neq j$ выполнено либо $r_i < l_j$, либо $r_j < l_i$.

\OutputFile
В единственной строке выведите одно целое число~--- минимальное количество отрезков, получившихся после операций расширения.

\Examples

\begin{example}
\exmpfile{example.01}{example.01.a}%
\exmpfile{example.02}{example.02.a}%
\end{example}

\Note
В первом примере за 2 операции можно расширить первый отрезок на 2 влево. Тогда отрезки объединятся в один: $[1,5]$.

Во втором примере можно расширить второй отрезок вправо на 1. Второй и третий отрезки объединятся в один. Получим 2 отрезка: $[1,2]$ и $[5,10]$. Получить один отрезок с помощью единственной оставшейся операции расширения невозможно.

\end{problem}

