\begin{tutorial}{Кроссворд}

\medskip
\textit{Авторы задачи: Азат Сафиуллин и Михаил Густокашин, разработка: Азат Сафиуллин}
\medskip

Данная задача может быть решена при помощи полного перебора. Для начала переберем, какое слово будет верхним, нижним, левым и правым. Всего существует $24$ возможных варианта. Далее для каждого варианта проверим, можно ли построить кроссворд.

Для того чтобы выполнить проверку, можно перебрать в каждом слове позиции $i$ и $j$ ($i < j$), в которых данное слово будет пересекаться с двумя другими. После этого необходимо проверить, совпадают ли символы на пересечении слов, а также совпадают ли длины противоположных сторон <<прямоугольника>>, который образуется в центре кроссворда. Данную проверку можно выполнить за $\mathcal{O}(n^8)$, где $n = \max\limits_{i=1}^{4} \lvert s_i \rvert$.

Таким образом, получаем решение за $\mathcal{O}(n^8)$ с константой порядка $24$. Данное решение может уложиться в ограничение при аккуратной реализации.

Также решение можно легко ускорить и получить оценку $\mathcal{O}(n^6)$ с константой порядка $24$, если для нижнего и правого слов перебирать только одну позицию (вторая позиция вычисляется по простой формуле, так как известна первая позиция и длина стороны <<прямоугольника>>).

\end{tutorial}
