\begin{problem}{Кроссворд}{стандартный ввод}{стандартный вывод}{1 секунда}{256 мегабайт}

Вам даны четыре различных слова. Из них необходимо составить кроссворд с двумя горизонтальными и двумя вертикальными словами так, чтобы выполнялись следующие условия:

\begin{itemize}
\item \textbf{Каждое} горизонтальное слово пересекается с \textbf{каждым} вертикальным;
\item Горизонтальные слова располагаются в разных строках;
\item Вертикальные слова располагаются в разных столбцах;
\item Все слова должны быть использованы ровно \textbf{один} раз;
\item Горизонтальные слова в кроссворде читаются слева направо, вертикальные~--- сверху вниз.
\end{itemize}

Для лучшего понимания задачи смотрите примеры.

\InputFile
В четырех строках записаны четыре различных слова $s_1$, $s_2$, $s_3$, $s_4$ ($2 \le \lvert s_i \rvert \le 10$). Каждое слово состоит из строчных букв латинского алфавита.

\OutputFile
В первой строке выведите <<\texttt{Yes}>>, если возможно составить кроссворд, удовлетворяющий требованиям из условия, или <<\texttt{No}>> в противном случае.

Если составить кроссворд возможно, выведите его в виде сетки $18 \times 18$. В пустые ячейки следует записывать символ <<\texttt{.}>> (точка), в непустые~--- строчные латинские буквы.

Если существует несколько возможных кроссвордов, выведите любой.

\Examples

\begin{example}
\exmpfile{example.01}{example.01.a}%
\exmpfile{example.02}{example.02.a}%
\exmpfile{example.03}{example.03.a}%
\end{example}

\end{problem}

