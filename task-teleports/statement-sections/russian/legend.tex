В Иннополисе проводятся гонки дронов.

В гонке могут принять участие $n$ дронов, $i$-й дрон пролетает единицу расстояния за $t_i$ секунд.
Гонка проводится на прямой, на которой расположены $m$ ворот, пронумерованных от $1$ до $m$, $i$-е ворота находятся на расстоянии $s_i$ от стартовой позиции гонки.

В гонке примут участие первые $k$ дронов с номерами от $1$ до $k$. Величину $k$ судьи объявляют непосредственно перед гонкой, поэтому вам необходимо проанализировать гонку для всех $k$ от~$1$~до~$n$.

Гонка проводится следующим образом. 

Дроны начинают движение из точки $0$ в сторону ворот, каждый со своей скоростью. У каждого дрона есть \textit{точка восстановления} "--- последние ворота, в которых он выполнял \textit{сохранение позиции}. Изначально точка восстановления каждого дрона "--- точка $0$. Дроны каждый раз начинают двигаться из своих точек восстановления и продолжают движение, пока один или несколько дронов не оказываются в точке, где расположены ворота (возможно, различные для разных дронов). В этот момент среди всех дронов, которые оказались в каких-либо воротах, выбирается дрон с наименьшим номером. Для этого дрона производится сохранение позиции, его точка восстановления переносится в его текущую позицию. Остальные дроны мгновенно телепортируются в свои точки восстановления. После этого гонка продолжается таким же образом. 

Как только дрон сохраняет позицию в последних воротах с номером $m$, он финиширует. Не финишировавшие пока дроны, как обычно, телепортируются в свои точки восстановления и продолжают гонку. Когда все дроны финишируют, гонка завершается.

Телепортация "--- очень энергоемкий процесс. Для подготовки к гонке необходимо понять, сколько суммарно телепортаций совершат все дроны до её завершения. Обозначим как $c_k$ суммарное число телепортаций, которое совершат все дроны, если в гонке будут участвовать первые $k$ дронов. Найдите значения $c_1, c_2, \ldots, c_n$.
