\ifdefined\tutorialbox \tutorialbox{Азат Сафиуллин, Иван Сафонов}{Иван Сафонов, Азат Сафиуллин} \fi

В первой подзадаче $n=2$. Достаточно пройтись двумя указателями по двум массивам, двигая указатель того дрона, который пролетает следующее расстояние быстрее (учитывая индексы дронов). $O(m)$

Во второй подзадаче достаточно запустить гонку для каждого префикса $k$. Заводим $k$ указателей и двигаем указатель того дрона, который пролетает отрезок быстрее остальных (учитывая индексы). Получаем решение за $O(n^3 m)$.

Для решения третьей подзадачи достаточно оптимизировать поиск дрона, который пролетит следующий отрезок быстрее всех. Это можно сделать структурой $Set$. $O(n^2 m \log(n))$

В четвертой подзадаче расстояния между точками равны. Очевидно, что первым в гонке все расстояния подряд пролетит дрон с минимальной скоростью. Вторым пролетит дрон с минимальной скоростью из оставшихся и так далее. Таким образом, ответ можно вычислить формулой $\frac{(k-1) k m}{2}$, где $k$ - текущий префикс. 

В пятой подзадаче все скорости равны. Зафиксируем $p$ "--- индекс левого вхождения максимального расстояния между ближайшими точками. Очевидно, что все отрезки левее $p$ имеют длину меньше. Поэтому в гонке будет момент, когда все дроны соберутся в точке $p$. Поскольку никакой дрон не может пролететь отрезок максимальной длины быстрее, чем какой-то дрон, который находится левее $p$. Понятно, что, как только какой-то дрон пролетит максимальное расстояние, он будет лететь до конца непрерывно. А это эквивалентно решению подзадачи $4$. То ответ для префикса $k$ будет равен $(p-1) k (k-1) + \frac{(k-1) k (m-p+1)}{2}$.

Обратим внимание на рекорды слева направо в строго возрастающем порядке. Нетрудно понять, что при преодолении дроном какого-то рекорда, он же продолжит непрерывно лететь до следующего, поскольку до следующего рекорда будут встречаться отрезки не больше текущего рекорда. Таким образом, можно <<сжать>> количество отрезков, определив за каждым рекордом количество не больших отрезков за ним. Нетрудно доказать, что количество рекордов не больше $\sqrt{s_m}$.


В шестой подзадаче $n \le 100$. Сожмем в рекорды. Запустим решение подзадачи $3$. Получаем решение за $O(n^2 \sqrt{s_m} \log(n))$.

В седьмой подзадаче $t_i \le 2$. Будем поддерживать ответ на префиксе. Когда добавляется новый дрон, достаточно добавить к ответу количество телепортаций, которое совершит этот дрон и количество вынужденных телепортаций других дронов, которые были созданы новым дроном. Это нетрудно посчитать, разобрав случаи $t_i = 1$ или $t_i = 2$. 

Пусть $d_1 = s_1$ и $d_i = s_i-s_{i-1}$ (для каждого $1<i$)

Поддерживаем ответ. Будем его обновлять после добавления нового дрона $i$.

Есть два случая:

\begin{enumerate}
\item когда другие вынуждают телепортироваться $i$.
\item когда $i$ вынуждает телепортироваться других.
\end{enumerate}

Переберем рекорд $u$ и будем искать подходящие под условия дроны

\begin{enumerate}
\item $t_i\cdot d_m \geq t_j\cdot d_u$, т.е. ищем такие $j$, что $i$-й еще не финишировал (эквивалентно непрохождению $i$-м последнего рекорда).
\item $t_i\cdot d_u < t_j\cdot d_m$, т.е. ищем такие $j$, что еще не финишировали (эквивалентно непрохождению $j$-м последнего рекорда).
\end{enumerate}

Такие $j$ можно искать корневой декомпозицией. Добавлений не больше $O(n)$, а количество запросов не больше $O(n\cdot \sqrt{s_m})$. Таким образом, добавление хотим выполнить за $O(\sqrt{t_m})$, а запрос $O(1)$. Получаем решение за $O(n (\sqrt{s_m} + \sqrt{maxv}))$.

Данная идея с реализацией проходит $11$ подзадач. 

Для полного решения давайте отсортируем значения $t_{p_1} \leq t_{p_2} \leq \ldots \leq t_{p_n}$. Будем перебирать индексы в порядке возрастания значений $t$, то есть перебирать $p_i$. Тогда для всех $1 \leq u \leq m$ будем поддерживать указатели $ptr_1[u]$ --- максимальное значение $j$, такое что $t_{p_i}\cdot d_m \geq t_{p_j}\cdot d_u$ и $ptr_2[u]$ --- минимальное значение $j$, такое что $t_{p_i}\cdot d_u < t_{p_j}\cdot d_m$. Поскольку мы перебираем значения в порядке возрастания, мы можем двигать указатели.

Посмотрим как меняется ответ в момент времени $p_i$. В неравенствах, написанных выше участвуют все индексы $p_j$ по $j \leq ptr_1[u]$ или $j \geq ptr_2[u]$ по всем $1 \leq u \leq m$. Но среди них нужно оставить только $p_j < p_i$. Давайте сделаем массив длины $n$, где в индексе $x$ будем хранить добавку, которую дает элемент с индексом $x$. В этой структуре данных нужно $O(n \sqrt{s_m})$ прибавлять к индексу (при движениях указателей) и $n$ раз считать сумму на префиксе. Тогда если применить для этого массива корневую декомпозицию, получится полное решение.

Мы нигде не использовали, что $t_i$ маленькие. Время работы $O(n (\sqrt{s_m} + \sqrt{n}))$, память $O(n)$.
